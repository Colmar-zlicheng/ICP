\documentclass{article}
\usepackage[left=1 in,top=0.5 in,right=1 in,bottom=1in]{geometry}
\usepackage{amsmath}

\title{ICP using SVD}

\begin{document}
\maketitle{}
\par
Define points sets:
\begin{equation}
    P = \left\{p_1, p_2, \ldots, p_n \right\}, \quad P' = \left\{p'_1, p'_2, \ldots, p'_n \right\}\\
\end{equation}
\par
Points transform:
\begin{equation}
\forall i, p_i = Rp'_i + t
\end{equation}
\par
Define error:
\begin{equation}
e_i = p_i - (Rp'_i + t)
\end{equation}
\begin{equation}
E = \frac{1}{N}\sum^{N}_{i=1}||p_i - (Rp'_i + t)||^2
\end{equation}
\par
Constructing least squares problems:
\begin{equation}
\mathop{\arg\min}\limits_{R,t}\frac{1}{N}\sum^{N}_{i=1}||p_i - (Rp'_i + t)||^2
\end{equation}
\par
Solve for the displacement matrix t:
\begin{equation}
\frac{dE}{dt}=\frac{2}{N}\sum^{N}_{i=1}(Rp'_i + t -p_i)=\frac{2}{N}\sum^{N}_{i=1}t+\frac{2}{N}\sum^{N}_{i=1}Rp'_i - \frac{2}{N}\sum^{N}_{i=1}p_i=0
\end{equation}

\begin{equation}
t+\frac{1}{N}\sum^{N}_{i=1}Rp'_i - \frac{1}{N}\sum^{N}_{i=1}p_i=0
\end{equation}
\par
Define centroid:
\begin{equation}
p = \frac{1}{N}\sum_{i=1}^{N}p_i, \quad p' = \frac{1}{N}\sum_{i=1}^{N}p'_i
\end{equation}
\par
Thus:
\begin{equation}
t = p - Rp'
\end{equation}
\par
Substitute back to (4)
\begin{align}
E&=\frac{1}{N}\sum^{N}_{i=1}||p_i - Rp'_i -p+Rp'||^2\\
&=\frac{1}{N}\sum^{N}_{i=1}||p_i - p- R(p'_i -p')||^2
\end{align}
\par
Define decentroided point clouds:
\begin{equation}
q_i=p_i-p, \quad q'_i=p'_i-p'
\end{equation}
\par
Thus:
\begin{gather}
    E = \frac{1}{N}\sum^{N}_{i=1}||q_i - Rq'_i||^2\\
    \mathop{\arg\min}\limits_{R}\frac{1}{N}\sum^{N}_{i=1}||q_i - Rq'_i||^2
\end{gather}
\par
Solve for the rotation matrix R:
\begin{equation}
||q_i - Rq'_i||^2 = (q_i - Rq'_i)^T(q_i - Rq'_i) = q_i^Tq_i - q_i^TRq'_i - q_i^{\prime T}R^Tq_i + q_i^{\prime T}R^TRq'_i
\end{equation}
\par
Since:
\begin{gather}
    q_i^TRq'_i: (1 \times 3)(3 \times 3)(3 \times 1)=(1 \times 1) \Rightarrow scalar \quad quantity\\
    q_i^{\prime T}R^Tq_i: (1 \times 3)(3 \times 3)(3 \times 1)=(1 \times 1) \Rightarrow scalar \quad quantity
\end{gather}
\par
Thus:
\begin{equation}
    ||q_i - Rq'_i||^2 = q_i^Tq_i - 2q_i^TRq'_i + q_i^{\prime T}q'_i
\end{equation}
\par
Substitute back to (13, 14)
\begin{equation}
    E = \frac{1}{N}\sum^{N}_{i=1}(q_i^Tq_i - 2q_i^TRq'_i + q_i^{\prime T}q'_i)
\end{equation}
\begin{align}
    &\mathop{\arg\min}\limits_{R}\frac{1}{N}\sum^{N}_{i=1}(q_i^Tq_i - 2q_i^TRq'_i + q_i^{\prime T}q'_i)\\
    =&\mathop{\arg\min}\limits_{R}\frac{1}{N}\sum^{N}_{i=1}(- 2q_i^TRq'_i)\\
    =&\mathop{\arg\max}\limits_{R}\frac{1}{N}\sum^{N}_{i=1}(q_i^TRq'_i)
\end{align}
\par
The trace of a constant is known to be equal to the constant itself, and tr(AB) = tr(BA):
\begin{align}
\frac{1}{N}\sum^{N}_{i=1}(q_i^TRq'_i)&=tr(\frac{1}{N}\sum^{N}_{i=1}(q_i^TRq'_i))\\
&=\frac{1}{N}tr(R\sum^{N}_{i=1}(q'_iq_i^T))
\end{align}
\par
Define:
\begin{equation}
    H = \sum^{N}_{i=1}(q_iq_i^{\prime T})\quad or \quad H = \frac{1}{N}\sum^{N}_{i=1}(q_iq_i^{\prime T})
\end{equation}
\par
Substitute back to (22)
\begin{equation}
    \mathop{\arg\max}\limits_{R}tr(RH^T)
\end{equation}
\par
SVD:
\begin{equation}
    H = U\Sigma V^T
\end{equation}
\begin{align}
\mathop{\arg\max}\limits_{R}tr(R(U\Sigma V^T)^T)&=\mathop{\arg\max}\limits_{R}tr(RV\Sigma U^T)\\
&=\mathop{\arg\max}\limits_{R}tr(\Sigma U^TRV)
\end{align}
\par
Define:
\begin{equation}
M=U^TRV
\end{equation}
\par
Since $R, U^T, V$ are orthogonal matrixs: $M$ is a special orthogonal matrix:
\begin{equation}
    \forall i,j, \quad |m_{ij}|<1, \quad  m \in M
\end{equation}

\begin{equation}
    \Sigma=
\begin{pmatrix}
    \sigma_1 &  & \\
     & \sigma_2 & \\
    &  & \sigma_3
\end{pmatrix}
,\quad \sigma_1, \sigma_2, \sigma_3 > 0
\end{equation}

\begin{align}
    tr(\Sigma M)&=
\begin{pmatrix}
    \sigma_1 &  & \\
     & \sigma_2 & \\
    &  & \sigma_3
\end{pmatrix}
\begin{pmatrix}
    m_{11} & m_{12} & m_{13} \\
    m_{21} & m_{22} & m_{23} \\
    m_{31} & m_{32} & m_{33}
\end{pmatrix}
\\
&=\sum_{i}\sigma_im_{ii}\\
&\leq \sum_{i}\sigma_i
\end{align}

\par
Take the equal sign (maximum value) at $m_{ii}=1$,
since M is a identity matrix:
\begin{equation}
    M=
\begin{pmatrix}
    1 &  & \\
     & 1 & \\
    &  & 1
\end{pmatrix}
=I
\end{equation}
\par
Thus:
\begin{gather}
    M=U^TRV=I\\
    R = UV^T\\
    t= p-Rp'=p-UV^Tp'
\end{gather}
% If you use vscode on MacOS, you may need the following lines:
% \cite{*}
% \bibliographystyle{IEEEtran}
% \bibliography{IEEEexample}
\end{document}